%%%%%%%%%%%%%%%%%%%%%%%%%%%%%%%%%%%%%%%%%%%%%%%%%%%%%%%%%%%%%%%%%%%%%%%%%%%%%%%
%2345678901234567890123456789012345678901234567890123456789012345678901234567890
%        1         2         3         4         5         6         7         8

%\documentclass[letterpaper, 10 pt, conference]{ieeeconf}  % Comment this line out
                                                          % if you need a4paper
\documentclass[a4paper, 10pt, conference]{ieeeconf}      % Use this line for a4
                                                          % paper
\usepackage[utf8]{inputenc}  % Ng Edit for accents in spanish
\usepackage[spanish]{babel}  % Ng Edit for accents in spanish
\usepackage{hyperref}        % Ng Edit for adding urls
\usepackage{graphicx}        % Ng Edit for adding graphics
\IEEEoverridecommandlockouts                              % This command is only
                                                          % needed if you want to
                                                          % use the \thanks command
\overrideIEEEmargins
% See the \addtolength command later in the file to balance the column lengths
% on the last page of the document

\def\equationautorefname~#1\null{(#1)\null}%to use parenthesis in eqs.


% The following packages can be found on http:\\www.ctan.org
%\usepackage{graphics} % for pdf, bitmapped graphics files
%\usepackage{epsfig} % for postscript graphics files
%\usepackage{mathptmx} % assumes new font selection scheme installed
%\usepackage{times} % assumes new font selection scheme installed
%\usepackage{amsmath} % assumes amsmath package installed
%\usepackage{amssymb}  % assumes amsmath package installed

\title{\LARGE \bf
    Utilización de Algoritmo de Agrupación de Múltiples Vistas
    para la Detección de Rostros
}
\author{Pedro Luis González Roa}

\begin{document}
    \maketitle
    \thispagestyle{empty}
    \pagestyle{empty}

    \begin{abstract}
        En los últimos años se ha demostrado la complejidad en la realización de
        algoritmos de detección y reconocimiento de rostros, ya que estos comparten
        un alto grado de similitud estructural. De esta manera, se requieren de la
        utilización \textit{redes neuronales convolucionales} (CNN por sus siglas
        en inglés) para la extración de características de la imagen.
        Las cuales requieren de una base de datos de gran tamaño para el
        entrenamiento exitoso de estas. Por lo que se utilizan se utilizan
        \textit{CNN}'s previamente entrenadas para la extracción de las características
        las cuales definirán el grupo ó cara a la que pertenecen. Se propone la
        utilización de múltiples \textit{redes neuronales convolucionales} en conjunto,
        combinando las características que extraen cada una de ellas para la parte
        interior y la exterior de la cara.
    \end{abstract}

    \section{Antecedentes}

    \subsection{Agrupamiento de imágenes}
    El reconocimiento facial se encuentra directamente relacionado con uno de los
    problemas que han recibido mucha atención en las últimas tres décadas. El problema
    de agrupamiento de imágenes, también conocido como \textit{IC} por sus siglas
    en inglés, se centra en la búsqueda y comparación de características similares
    entre imágenes. En otras palabras, \textit{IC} consiste subconjuntos de objetos de
    una misma naturaleza y separarlos de los objetos con características diferentes.
    \cite{CombiningCNN}

    \subsection{Combinación de CNN previamente entrenados}
    En previas investigaciones diferentes modelos de \textit{deep clustering algorithm}
    han demostrado un buen desempeño para pequeñas imágenes. Por otro lado, para
    obtener resultados exitosos en la agrupación de imágenes de mayor complejidad
    (cómo imágenes de objetos con estructuras complejas) es necesario la utilización
    de \textit{CNN}s ya entrenadas en un paso previo para la extracción de
    características específicas. Existen una variedad de estos modelos, los cuales
    tienen un mismo objetivo pero aportan una perspectiva diferente. Para casos
    específicos un modelo puede tener mejor desempeño en la obtención de
    características definitivas, pero en otros existe la posibilidad de lo
    contrario. \cite{CombiningCNN}

    \newline
    Al igual que en diferentes temas en la era de \textit{Big Data}, existen
    diferentes perspectivas del mismo objeto que se pueden utilizar para complementarse
    mutuamente. Por lo cual, el algoritmo de
    \textit{Agrupación de vistas múltiples (Multi-view clustering - MVC)} ha ganado
    popularidad por su éxito en modelos que obtienen información del mismo objeto
    desde diferentes fuentes. \cite{Yang2018} Guérin en su artículo
    \textit{"Combining pretrained CNN feature extractors to enhance clustering of complex natural images"}
    utiliza diferentes modelos \textit{CNN} cómo las múltiples perspectivas, para obtener
    un mejor rendimiento en la agrupación de imágenes. Utilizando dicha arquitectura
    se demostró que se reduce el riesgo de elegir una red \textit{CNN} que no
    obtenga las mejores características para la discriminación de las imágenes.
    \cite{CombiningCNN}


    \subsection{Modelo IE-CNN}
    Previamente se mencionó que el reconocimiento facial se encuentra relacionado
    a \textit{IC}. Esto es porque también es importante la detección de los rasgos
    que diferencían a una persona de otra; donde por la estructura general de nuestros
    rostros no es una tarea sencilla. Por lo que investigaciones y propuestas exitosas
    en este ámbito han utilizado una gran cantidad de imágenes de los mismos rostros.
    An-Ping Song en su artículo \textit{"Similar Face Recognition Using the IE-CNN Model"}
    remarca que en estas investigaciones se ha enfocado a la cara como un solo objeto.
    Cuando nuestras cabezas se descomponen en dos áreas:
    \begin{itemize}
        \item La interna: que se compone por los ojos, nariz y boca.
        \item La externa: que se compone por la cabeza, la barbilla y las orejas.
    \end{itemize}
    El modelo \textit{IE-CNN} de Song se enfoca en separar dichas áreas y utilizarlas
    cómo diferentes partes de una misma entrada de datos antes de las estructuras
    convolucionales, obteniendo mejores resultados. \cite{IECNN}

    \section{Planteamiento del Problema}
    La utilización de redes neuronales profundas ha obtenido un buen resultado
    para la detección de rostros. Aunque un atributo importante para el éxito de
    estas ha sido el enorme tamaño de las bases de datos utilizadas para el
    entrenamiento. Esto es por la similitudes estructurales que se tienen entre
    los diferentes rostros humanos, exigiendo una gran cantidad de muestras para
    que un modelo \textit{CNN} convencional sea capaz de aprender
    sobre las características necesarias para discernir entre ellas.

    \newline
    La obtención de dicha cantidad de datos requiere de un personal extenso y acceso
    a mucha información, la cual siempre es pública. An-Ping Song en su
    investigación describe la gran diferencia entre los repositorios públicos y
    los repositorios privados de rostros humanos para entrenamiento en la
    tabla \textit{Table 1}. \cite{IECNN} Al no ser tan viable la recolección de
    estos datos, en ocasiones es necesario buscar alternativas para eficientizar
    los modelos alrededor de la cantidad de información obtenida.
    
    \begin{table}[ht]
        \caption{Tamaño de diferentes repositorios de imágenes de rostros}
        \label{t1}
        \begin{center}
            \begin{tabular}{|c||c||c|}
                \hline
                \textbf{Repositorio} & \textbf{Identidades} & \textbf{Imágenes} \\
                \hline
                LFW & 5,749 & 13,233 \\
                \hline
                WDRef & 2,995 & 13,233 \\
                \hline
                CelebFaces & 10,177 & 202,599 \\
                \hline
                CASIA-WebFace & 10,575 & 494,414 \\
                \hline
                Facebook & 4,030 & 4.4M \\
                \hline
                Google (Privado) & 8M & 200M \\
                \hline
            \end{tabular}
        \end{center}
    \end{table}

    \section{Contribuciones Esperadas}
    Se propone un modelo basado en los resultados de las investigaciones previamente
    mencionadas. Esta arquitectura buscará utilizar diferentes perspectivas de
    modelos \textit{CNN} previamente entrenados sobre las dos diferentes áreas
    propuestas por Song, utilizando agrupamiento de diferentes vistas. Buscando
    aprovechar al máximo la cantidad de información en los dataset públicos, y
    tener el mismo desempeño, o incluso una mejora, a comparación de otros modelos.

    \newline
    El desempeño del modelo se mide de acuerdo al porcentage de rostros correctamente
    agrupados. Es decir que los grupos de las imágenes correspondan al mismo rostro.

    \bibliographystyle{plain} % We choose the "plain" reference style
    \bibliography{refs} % Entries are in the refs.bib file
\end{document}
